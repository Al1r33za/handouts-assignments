\documentclass[a4paper,twocolumn]{article}
\usepackage{amsmath}
\usepackage{listings}

\title{random process assignment 1}
\author{Alireza Mousazadeh}

\newcommand{\dd}[1]{\frac{\mathrm{d}}{\mathrm{d #1}}}

\begin{document}
	\maketitle
	
	\section{One}
	\(X \sim Binom(n, k); n=2000, k=600\)
	\[\Pr(k=600) = \binom{2000}{600} \frac{1}{2}^{600} \frac{1}{2}^{1400}\]
	\section{Two}
	\(PDF: f_X(x) = A\exp(-x)u(x), X \sim f_X(x)\)
	\begin{align}
		\int_{-\infty}^{\infty}f_X(x') dx' &= 1 \nonumber \\
		\int_{-\infty}^{\infty}Ae^{-x'}u(x')dx'&=1\nonumber \\
		\int_{0}^{\infty}Ae^{-x'} dx'&=1\nonumber \\
		\left. -Ae^{-x'} \right|_0^\infty = A &= 1
	\end{align}
	\begin{align}
		F_X(x) = \int_0^x e^{-x'} dx' &= \left. -e^{-x'} \right|_0^x = 1 - e^{-x} 
	\end{align}
	\begin{align}
		\Pr\{1<x<2\} &= \Pr\{x>1, x<2\} \nonumber \\
		=F_X(2) - F_X(1) &= e^{-1} - e^{-2} 
	\end{align}
	
	\section{Three}
	\(f_{XY}(x,y) = A\exp(-(x+y))u(x)u(y)\)
	
	\begin{align}
		\iint_{\mathcal{X,Y}} f_{XY}(x,y) dxdy &= 1 \nonumber \\
		\iint_{0}^{\infty} Ae^{-(x+y)} dxdy &= A\int_{0}^{\infty} e^{-x} \int_{0}^{\infty} e^{-y} dxdy =1 \nonumber \\
		&= A = 1
	\end{align}
	\begin{align}
		f_{X|Y}(x|y) = \frac{f_{XY}}{f_Y} = f_X\nonumber \\
		f_{XY} = f_Y . f_X \nonumber \\
		i \quad f_X = \int_{\mathcal{Y}} f_{XY} = e^{-x} \nonumber \\
		ii \quad f_Y =\int_{\mathcal{X}} f_{XY} = e^{-y} \nonumber \\
		i, ii \to f_{XY} = e^{-x}.e^{-y} = e^{-(x+y)} \quad \diamond
	\end{align}
	$\to X,Y$ are independent.
	\begin{align}
		\to & f_{X|Y} = f_X = e^{-x} \nonumber \\
		& f_{Y|X} = f_Y = e^{-y} \nonumber 
	\end{align}
	
	\section{Four}
	\(Y = g(x)= a\sin(X+\theta) \) let solutions be $x_n$ :
	\[g(x_n) = y \quad g'(x_n) = a\cos(x_n+\theta)\] 
	\[\cos(x_n+\theta) =\sqrt{1 - \sin^2(x_n+\theta)} = \sqrt{1 - (\frac{y}{a})^2}\] 
	\[\to g(x_n) = \frac{1}{a} \sqrt{a^2-y^2} \to g'(x_n) = \sqrt{a^2-y^2}\]
	\[\to f_Y = \sum_n\left(\frac{f_X(x_n)}{\sqrt{a^2-y^2}}\right)\]
	and if $X \sim Uni(-\pi, \pi)$ then $g(x_n) = y$ has two solution over $(-\pi,\pi)$ for $-4 <y < 4$:
	\[f_Y = \left(\frac{1}{2\pi\sqrt{16-y^2}}\right) + \left(\frac{1}{2\pi\sqrt{16-y^2}}\right) = \left(\frac{2}{2\pi\sqrt{16-y^2}}\right)\]
	
	\section{Five}
	\(X \sim P(a) \;;\; P(X = k) = \exp(-a)a^k/k!\to E_x,\sigma_X^2 = ?\)
	\begin{align*}
		E[X] = \sum_{k=1}^{\infty} k\frac{e^{-a}a^k}{k!} \\	
		e^a = \sum_{k=0}^{\infty} \frac{a^k}{k!} =
		\dd{a}e^a = \frac{1}{a}\sum_{k=1}^{\infty}k \frac{a^{k}}{k!} \to E[X] = a \quad \diamond\\
		\dd{a} \big(\dd{a}\big) = \frac{1}{a^2}\sum_{k=1}^{\infty}k(k-1) \frac{a^{k}}{k!} \to E[k^2] = a^2 + a\\
		\to \sigma_X^2 = E[k^2] - E^2[k]= a^2\quad \diamond
	\end{align*}
	
	\section{Six}
	\(\Phi_X = e^{j\mu\omega-1/_2\sigma^2\omega^2} = E[e^{j\mu\omega}]\) defining new r.v. $Z$:
	\[X = \mu + \sigma Z \;;\; Z \sim \mathcal{N}(0,1)\]
	\[\Phi_X(\omega) = e^{j\mu\omega}.E[e^{j\sigma\omega z}] = e^{j\mu\omega}.\Phi_Z(\sigma\omega)\]
	\begin{align*}
		\Phi_Z(t) = \frac{1}{\sqrt{2\pi}}\int e^{jtz}.e^{-z^2/2}\; dz \\
		= \frac{1}{\sqrt{2\pi}}\int e^{(jtz-z^2/2)} \; dz\\
		= \frac{1}{\sqrt{2\pi}}\int e^{-\frac{1}{2}((z-jt)^2+t^2)}\; dz\\
		= \frac{e^{-t^2/2}}{\sqrt{2\pi}}\int e^{-\frac{1}{2}(z-jt)^2}\; dz = e^{\frac{-t^2}{2}}\\
		\to \Phi_X(\omega) = e^{j\mu\omega}.e^{-\frac{(\sigma\omega)^2}{2}} \quad \diamond
	\end{align*}
	
	\section{Seven}
	because $X$ and $Y$ are normal, $Z$, $W$ and their joint distribution are normal too. and we have the following:
	\[\mu_z = 10 + 0 = 10 \]
	\[\mu_w = 10 - 0 = 10\]
	\[\sigma_z^2 = 4 + 1 -2(.5)\]
	\[\sigma_z^2 = 4 + 1 -2(.5)\]
	\[C_{zw} = C_{x+y, x-y} = Var(x) - Var(y) = 4 - 1 =3\]
	covariance matrix: \(\to \Sigma_{zw} = \begin{pmatrix}
		6 & 3\\
		3 & 4
	\end{pmatrix},\; \mu_z = 10,\; \mu_w = 10\)
	\[\det(\Sigma_{zw}) = 24 - 9 = 15 \to \Sigma_{zw}^{-1} =\frac{1}{15}\begin{pmatrix}
		4 & -3\\
		-3 & 6
	\end{pmatrix}\]
	\[\to f_{zw} = \frac{1}{2\pi\sqrt{15}}\exp\Big((-1/(2.15)
		\begin{pmatrix}
			z - 10 & w - 10
		\end{pmatrix}.\begin{pmatrix}
		4 & -3\\
		-3 & 6
		\end{pmatrix}.\begin{pmatrix}
		z-10\\
		w-10
		\end{pmatrix}\Big)\]
		
	\section{Eight}
	following code generates a gif describing the central limit theory:
	\begin{lstlisting}
		Nsim = 50000;
		
		n_max = 100;
		n_values = 1:n_max;
		
		gif_name = 'CLT_animation.gif';
		
		for n = n_values
		
		X = rand(Nsim, n);
		
		S = sum(X, 2);
		
		mu = n * 0.5;
		sigma = sqrt(n * 1/12);
		
		histogram(S, 'Normalization', 'pdf', 'BinWidth', 0.1); hold on;
		
		x = linspace(min(S), max(S), 300);
		y = 1/(sigma*sqrt(2*pi)) * exp(-(x-mu).^2/(2*sigma^2));
		plot(x, y, 'r', 'LineWidth', 2);
		
		title(['Central Limit Theorem  |  n = ' num2str(n)]);
		xlabel('S_n');
		ylabel('Density');
		xlim([mu - 5*sigma, mu + 5*sigma]);
		grid on;
		
		drawnow;
		
		frame = getframe(gcf);
		im = frame2im(frame);
		[A, map] = rgb2ind(im, 256);
		
		if n == 1
		imwrite(A, map, gif_name, 'gif', 'LoopCount', Inf, 'DelayTime', 0.1);
		else
		imwrite(A, map, gif_name, 'gif', 'WriteMode', 'append', 'DelayTime', 0.1);
		end
		
		hold off;
		end
	\end{lstlisting}
\end{document}































